% Generated by Sphinx.
\def\sphinxdocclass{report}
\documentclass[letterpaper,10pt,english]{sphinxmanual}
\usepackage[utf8]{inputenc}
\DeclareUnicodeCharacter{00A0}{\nobreakspace}
\usepackage{cmap}
\usepackage[T1]{fontenc}
\usepackage{babel}
\usepackage{times}
\usepackage[Bjarne]{fncychap}
\usepackage{longtable}
\usepackage{sphinx}
\usepackage{multirow}


\title{Proper Orthogonal Decomposition Documentation}
\date{May 14, 2014}
\release{0.0.5}
\author{Johannes Scharlach}
\newcommand{\sphinxlogo}{}
\renewcommand{\releasename}{Release}
\makeindex

\makeatletter
\def\PYG@reset{\let\PYG@it=\relax \let\PYG@bf=\relax%
    \let\PYG@ul=\relax \let\PYG@tc=\relax%
    \let\PYG@bc=\relax \let\PYG@ff=\relax}
\def\PYG@tok#1{\csname PYG@tok@#1\endcsname}
\def\PYG@toks#1+{\ifx\relax#1\empty\else%
    \PYG@tok{#1}\expandafter\PYG@toks\fi}
\def\PYG@do#1{\PYG@bc{\PYG@tc{\PYG@ul{%
    \PYG@it{\PYG@bf{\PYG@ff{#1}}}}}}}
\def\PYG#1#2{\PYG@reset\PYG@toks#1+\relax+\PYG@do{#2}}

\expandafter\def\csname PYG@tok@gd\endcsname{\def\PYG@tc##1{\textcolor[rgb]{0.63,0.00,0.00}{##1}}}
\expandafter\def\csname PYG@tok@gu\endcsname{\let\PYG@bf=\textbf\def\PYG@tc##1{\textcolor[rgb]{0.50,0.00,0.50}{##1}}}
\expandafter\def\csname PYG@tok@gt\endcsname{\def\PYG@tc##1{\textcolor[rgb]{0.00,0.27,0.87}{##1}}}
\expandafter\def\csname PYG@tok@gs\endcsname{\let\PYG@bf=\textbf}
\expandafter\def\csname PYG@tok@gr\endcsname{\def\PYG@tc##1{\textcolor[rgb]{1.00,0.00,0.00}{##1}}}
\expandafter\def\csname PYG@tok@cm\endcsname{\let\PYG@it=\textit\def\PYG@tc##1{\textcolor[rgb]{0.25,0.50,0.56}{##1}}}
\expandafter\def\csname PYG@tok@vg\endcsname{\def\PYG@tc##1{\textcolor[rgb]{0.73,0.38,0.84}{##1}}}
\expandafter\def\csname PYG@tok@m\endcsname{\def\PYG@tc##1{\textcolor[rgb]{0.13,0.50,0.31}{##1}}}
\expandafter\def\csname PYG@tok@mh\endcsname{\def\PYG@tc##1{\textcolor[rgb]{0.13,0.50,0.31}{##1}}}
\expandafter\def\csname PYG@tok@cs\endcsname{\def\PYG@tc##1{\textcolor[rgb]{0.25,0.50,0.56}{##1}}\def\PYG@bc##1{\setlength{\fboxsep}{0pt}\colorbox[rgb]{1.00,0.94,0.94}{\strut ##1}}}
\expandafter\def\csname PYG@tok@ge\endcsname{\let\PYG@it=\textit}
\expandafter\def\csname PYG@tok@vc\endcsname{\def\PYG@tc##1{\textcolor[rgb]{0.73,0.38,0.84}{##1}}}
\expandafter\def\csname PYG@tok@il\endcsname{\def\PYG@tc##1{\textcolor[rgb]{0.13,0.50,0.31}{##1}}}
\expandafter\def\csname PYG@tok@go\endcsname{\def\PYG@tc##1{\textcolor[rgb]{0.20,0.20,0.20}{##1}}}
\expandafter\def\csname PYG@tok@cp\endcsname{\def\PYG@tc##1{\textcolor[rgb]{0.00,0.44,0.13}{##1}}}
\expandafter\def\csname PYG@tok@gi\endcsname{\def\PYG@tc##1{\textcolor[rgb]{0.00,0.63,0.00}{##1}}}
\expandafter\def\csname PYG@tok@gh\endcsname{\let\PYG@bf=\textbf\def\PYG@tc##1{\textcolor[rgb]{0.00,0.00,0.50}{##1}}}
\expandafter\def\csname PYG@tok@ni\endcsname{\let\PYG@bf=\textbf\def\PYG@tc##1{\textcolor[rgb]{0.84,0.33,0.22}{##1}}}
\expandafter\def\csname PYG@tok@nl\endcsname{\let\PYG@bf=\textbf\def\PYG@tc##1{\textcolor[rgb]{0.00,0.13,0.44}{##1}}}
\expandafter\def\csname PYG@tok@nn\endcsname{\let\PYG@bf=\textbf\def\PYG@tc##1{\textcolor[rgb]{0.05,0.52,0.71}{##1}}}
\expandafter\def\csname PYG@tok@no\endcsname{\def\PYG@tc##1{\textcolor[rgb]{0.38,0.68,0.84}{##1}}}
\expandafter\def\csname PYG@tok@na\endcsname{\def\PYG@tc##1{\textcolor[rgb]{0.25,0.44,0.63}{##1}}}
\expandafter\def\csname PYG@tok@nb\endcsname{\def\PYG@tc##1{\textcolor[rgb]{0.00,0.44,0.13}{##1}}}
\expandafter\def\csname PYG@tok@nc\endcsname{\let\PYG@bf=\textbf\def\PYG@tc##1{\textcolor[rgb]{0.05,0.52,0.71}{##1}}}
\expandafter\def\csname PYG@tok@nd\endcsname{\let\PYG@bf=\textbf\def\PYG@tc##1{\textcolor[rgb]{0.33,0.33,0.33}{##1}}}
\expandafter\def\csname PYG@tok@ne\endcsname{\def\PYG@tc##1{\textcolor[rgb]{0.00,0.44,0.13}{##1}}}
\expandafter\def\csname PYG@tok@nf\endcsname{\def\PYG@tc##1{\textcolor[rgb]{0.02,0.16,0.49}{##1}}}
\expandafter\def\csname PYG@tok@si\endcsname{\let\PYG@it=\textit\def\PYG@tc##1{\textcolor[rgb]{0.44,0.63,0.82}{##1}}}
\expandafter\def\csname PYG@tok@s2\endcsname{\def\PYG@tc##1{\textcolor[rgb]{0.25,0.44,0.63}{##1}}}
\expandafter\def\csname PYG@tok@vi\endcsname{\def\PYG@tc##1{\textcolor[rgb]{0.73,0.38,0.84}{##1}}}
\expandafter\def\csname PYG@tok@nt\endcsname{\let\PYG@bf=\textbf\def\PYG@tc##1{\textcolor[rgb]{0.02,0.16,0.45}{##1}}}
\expandafter\def\csname PYG@tok@nv\endcsname{\def\PYG@tc##1{\textcolor[rgb]{0.73,0.38,0.84}{##1}}}
\expandafter\def\csname PYG@tok@s1\endcsname{\def\PYG@tc##1{\textcolor[rgb]{0.25,0.44,0.63}{##1}}}
\expandafter\def\csname PYG@tok@gp\endcsname{\let\PYG@bf=\textbf\def\PYG@tc##1{\textcolor[rgb]{0.78,0.36,0.04}{##1}}}
\expandafter\def\csname PYG@tok@sh\endcsname{\def\PYG@tc##1{\textcolor[rgb]{0.25,0.44,0.63}{##1}}}
\expandafter\def\csname PYG@tok@ow\endcsname{\let\PYG@bf=\textbf\def\PYG@tc##1{\textcolor[rgb]{0.00,0.44,0.13}{##1}}}
\expandafter\def\csname PYG@tok@sx\endcsname{\def\PYG@tc##1{\textcolor[rgb]{0.78,0.36,0.04}{##1}}}
\expandafter\def\csname PYG@tok@bp\endcsname{\def\PYG@tc##1{\textcolor[rgb]{0.00,0.44,0.13}{##1}}}
\expandafter\def\csname PYG@tok@c1\endcsname{\let\PYG@it=\textit\def\PYG@tc##1{\textcolor[rgb]{0.25,0.50,0.56}{##1}}}
\expandafter\def\csname PYG@tok@kc\endcsname{\let\PYG@bf=\textbf\def\PYG@tc##1{\textcolor[rgb]{0.00,0.44,0.13}{##1}}}
\expandafter\def\csname PYG@tok@c\endcsname{\let\PYG@it=\textit\def\PYG@tc##1{\textcolor[rgb]{0.25,0.50,0.56}{##1}}}
\expandafter\def\csname PYG@tok@mf\endcsname{\def\PYG@tc##1{\textcolor[rgb]{0.13,0.50,0.31}{##1}}}
\expandafter\def\csname PYG@tok@err\endcsname{\def\PYG@bc##1{\setlength{\fboxsep}{0pt}\fcolorbox[rgb]{1.00,0.00,0.00}{1,1,1}{\strut ##1}}}
\expandafter\def\csname PYG@tok@kd\endcsname{\let\PYG@bf=\textbf\def\PYG@tc##1{\textcolor[rgb]{0.00,0.44,0.13}{##1}}}
\expandafter\def\csname PYG@tok@ss\endcsname{\def\PYG@tc##1{\textcolor[rgb]{0.32,0.47,0.09}{##1}}}
\expandafter\def\csname PYG@tok@sr\endcsname{\def\PYG@tc##1{\textcolor[rgb]{0.14,0.33,0.53}{##1}}}
\expandafter\def\csname PYG@tok@mo\endcsname{\def\PYG@tc##1{\textcolor[rgb]{0.13,0.50,0.31}{##1}}}
\expandafter\def\csname PYG@tok@mi\endcsname{\def\PYG@tc##1{\textcolor[rgb]{0.13,0.50,0.31}{##1}}}
\expandafter\def\csname PYG@tok@kn\endcsname{\let\PYG@bf=\textbf\def\PYG@tc##1{\textcolor[rgb]{0.00,0.44,0.13}{##1}}}
\expandafter\def\csname PYG@tok@o\endcsname{\def\PYG@tc##1{\textcolor[rgb]{0.40,0.40,0.40}{##1}}}
\expandafter\def\csname PYG@tok@kr\endcsname{\let\PYG@bf=\textbf\def\PYG@tc##1{\textcolor[rgb]{0.00,0.44,0.13}{##1}}}
\expandafter\def\csname PYG@tok@s\endcsname{\def\PYG@tc##1{\textcolor[rgb]{0.25,0.44,0.63}{##1}}}
\expandafter\def\csname PYG@tok@kp\endcsname{\def\PYG@tc##1{\textcolor[rgb]{0.00,0.44,0.13}{##1}}}
\expandafter\def\csname PYG@tok@w\endcsname{\def\PYG@tc##1{\textcolor[rgb]{0.73,0.73,0.73}{##1}}}
\expandafter\def\csname PYG@tok@kt\endcsname{\def\PYG@tc##1{\textcolor[rgb]{0.56,0.13,0.00}{##1}}}
\expandafter\def\csname PYG@tok@sc\endcsname{\def\PYG@tc##1{\textcolor[rgb]{0.25,0.44,0.63}{##1}}}
\expandafter\def\csname PYG@tok@sb\endcsname{\def\PYG@tc##1{\textcolor[rgb]{0.25,0.44,0.63}{##1}}}
\expandafter\def\csname PYG@tok@k\endcsname{\let\PYG@bf=\textbf\def\PYG@tc##1{\textcolor[rgb]{0.00,0.44,0.13}{##1}}}
\expandafter\def\csname PYG@tok@se\endcsname{\let\PYG@bf=\textbf\def\PYG@tc##1{\textcolor[rgb]{0.25,0.44,0.63}{##1}}}
\expandafter\def\csname PYG@tok@sd\endcsname{\let\PYG@it=\textit\def\PYG@tc##1{\textcolor[rgb]{0.25,0.44,0.63}{##1}}}

\def\PYGZbs{\char`\\}
\def\PYGZus{\char`\_}
\def\PYGZob{\char`\{}
\def\PYGZcb{\char`\}}
\def\PYGZca{\char`\^}
\def\PYGZam{\char`\&}
\def\PYGZlt{\char`\<}
\def\PYGZgt{\char`\>}
\def\PYGZsh{\char`\#}
\def\PYGZpc{\char`\%}
\def\PYGZdl{\char`\$}
\def\PYGZhy{\char`\-}
\def\PYGZsq{\char`\'}
\def\PYGZdq{\char`\"}
\def\PYGZti{\char`\~}
% for compatibility with earlier versions
\def\PYGZat{@}
\def\PYGZlb{[}
\def\PYGZrb{]}
\makeatother

\begin{document}

\maketitle
\tableofcontents
\phantomsection\label{index::doc}


Contents:


\chapter{analysis module}
\label{analysis:analysis-module}\label{analysis:module-analysis}\label{analysis::doc}\label{analysis:welcome-to-s-documentation}\index{analysis (module)}\index{Timer (class in analysis)}

\begin{fulllineitems}
\phantomsection\label{analysis:analysis.Timer}\pysigline{\strong{class }\code{analysis.}\bfcode{Timer}}
Bases: \code{object}

Allows some basic profiling

\end{fulllineitems}

\index{plotResults() (in module analysis)}

\begin{fulllineitems}
\phantomsection\label{analysis:analysis.plotResults}\pysiglinewithargsret{\code{analysis.}\bfcode{plotResults}}{\emph{Y}, \emph{T}, \emph{label=None}, \emph{legend\_loc='upper left'}, \emph{show\_legend=False}}{}
\end{fulllineitems}

\index{randomRuns() (in module analysis)}

\begin{fulllineitems}
\phantomsection\label{analysis:analysis.randomRuns}\pysiglinewithargsret{\code{analysis.}\bfcode{randomRuns}}{\emph{sys}, \emph{rsys}, \emph{T}, \emph{sigma=10.0}, \emph{integrator='dopri5'}}{}
\end{fulllineitems}

\index{runAnalysis() (in module analysis)}

\begin{fulllineitems}
\phantomsection\label{analysis:analysis.runAnalysis}\pysiglinewithargsret{\code{analysis.}\bfcode{runAnalysis}}{\emph{n}, \emph{k}, \emph{N=1}, \emph{example='butter'}, \emph{T=20}, \emph{sigma=1.0}, \emph{integrator='dopri5'}}{}
\end{fulllineitems}

\index{x0() (in module analysis)}

\begin{fulllineitems}
\phantomsection\label{analysis:analysis.x0}\pysiglinewithargsret{\code{analysis.}\bfcode{x0}}{\emph{n}}{}
\end{fulllineitems}



\chapter{debug module}
\label{debug:debug-module}\label{debug::doc}\label{debug:module-debug}\index{debug (module)}\index{InputOfCalls (class in debug)}

\begin{fulllineitems}
\phantomsection\label{debug:debug.InputOfCalls}\pysiglinewithargsret{\strong{class }\code{debug.}\bfcode{InputOfCalls}}{\emph{f}}{}
Bases: \code{object}

Save the different calls to functions in order to see what might go wrong somewhere
\index{inputs() (debug.InputOfCalls class method)}

\begin{fulllineitems}
\phantomsection\label{debug:debug.InputOfCalls.inputs}\pysiglinewithargsret{\strong{classmethod }\bfcode{inputs}}{}{}
return a dict of \{function: {[}inputs{]},...\}

\end{fulllineitems}

\index{instances (debug.InputOfCalls attribute)}

\begin{fulllineitems}
\phantomsection\label{debug:debug.InputOfCalls.instances}\pysigline{\bfcode{instances}\strong{ = \{\}}}
\end{fulllineitems}


\end{fulllineitems}



\chapter{example2sys module}
\label{example2sys:example2sys-module}\label{example2sys::doc}\label{example2sys:module-example2sys}\index{example2sys (module)}\index{example2sys() (in module example2sys)}

\begin{fulllineitems}
\phantomsection\label{example2sys:example2sys.example2sys}\pysiglinewithargsret{\code{example2sys.}\bfcode{example2sys}}{\emph{filename}}{}
\end{fulllineitems}

\index{generateRandomExample() (in module example2sys)}

\begin{fulllineitems}
\phantomsection\label{example2sys:example2sys.generateRandomExample}\pysiglinewithargsret{\code{example2sys.}\bfcode{generateRandomExample}}{\emph{n, m, p=None, distribution=\textless{}bound method Random.gauss of \textless{}random.Random object at 0x7fd3ec11d620\textgreater{}\textgreater{}, distributionArguments={[}0.0, 1.0{]}}}{}
Generate a random example of arbitraty order
\begin{description}
\item[{How to use:}] \leavevmode
sys = generateRandomExample(n, m, {[}p, distribution{]})

\item[{Inputs:}] \leavevmode
n   system order
m   number of Inputs
p   number of outputs {[}p=m{]}
distribution
\begin{quote}

distribution of the matrix values {[}distribution=gauss(0., 1.){]}
\end{quote}

\item[{Output:}] \leavevmode
sys random StateSpaceSystem with the parameters set in the input

\end{description}

\end{fulllineitems}

\index{heatSystem() (in module example2sys)}

\begin{fulllineitems}
\phantomsection\label{example2sys:example2sys.heatSystem}\pysiglinewithargsret{\code{example2sys.}\bfcode{heatSystem}}{\emph{N}, \emph{L=1.0}, \emph{g0=0.0}, \emph{gN=0.0}}{}
\end{fulllineitems}

\index{optionPricing() (in module example2sys)}

\begin{fulllineitems}
\phantomsection\label{example2sys:example2sys.optionPricing}\pysiglinewithargsret{\code{example2sys.}\bfcode{optionPricing}}{\emph{N=None}, \emph{option='put'}, \emph{r=0.05}, \emph{T=1.0}, \emph{K=100.0}, \emph{L=None}}{}
\end{fulllineitems}

\index{stableRandomSystem() (in module example2sys)}

\begin{fulllineitems}
\phantomsection\label{example2sys:example2sys.stableRandomSystem}\pysiglinewithargsret{\code{example2sys.}\bfcode{stableRandomSystem}}{\emph{*args}, \emph{**kwargs}}{}
\end{fulllineitems}



\chapter{futurescipy module}
\label{futurescipy:module-futurescipy}\label{futurescipy::doc}\label{futurescipy:futurescipy-module}\index{futurescipy (module)}
Some functions that will be integrated into scipy in a future version. The use
of this file should be avoided as soon as the functions are fixed in scipy.
\index{abcd\_normalize() (in module futurescipy)}

\begin{fulllineitems}
\phantomsection\label{futurescipy:futurescipy.abcd_normalize}\pysiglinewithargsret{\code{futurescipy.}\bfcode{abcd\_normalize}}{\emph{A=None}, \emph{B=None}, \emph{C=None}, \emph{D=None}}{}
Check state-space matrices and ensure they are rank-2.

If enough information on the system is provided, that is, enough
properly-shaped arrays are passed to the function, the missing ones
are built from this information, ensuring the correct number of
rows and columns. Otherwise a ValueError is raised.
\begin{quote}\begin{description}
\item[{Parameters}] \leavevmode
\textbf{B, C, D} (\emph{A,}) -- State-space matrices. All of them are None (missing) by default.

\item[{Returns}] \leavevmode
\textbf{A, B, C, D} --
Properly shaped state-space matrices.

\item[{Return type}] \leavevmode
array

\item[{Raises}] \leavevmode
\code{ValueError} --
If not enough information on the system was provided.

\end{description}\end{quote}

\end{fulllineitems}



\chapter{pod module}
\label{pod:module-pod}\label{pod:pod-module}\label{pod::doc}\index{pod (module)}
Model order reduction for linear state space systems can be done with
Proper Orthogonal Decomposition (POD) methods. Some of them can be simply
applied when creating a system.
\index{isStable() (in module pod)}

\begin{fulllineitems}
\phantomsection\label{pod:pod.isStable}\pysiglinewithargsret{\code{pod.}\bfcode{isStable}}{\emph{A}}{}
\end{fulllineitems}

\index{lss (class in pod)}

\begin{fulllineitems}
\phantomsection\label{pod:pod.lss}\pysiglinewithargsret{\strong{class }\code{pod.}\bfcode{lss}}{\emph{*create\_from}, \emph{**reduction\_options}}{}
Bases: \code{object}

linear time independent state space system.

Default contstructor is called like \code{lss(A,B,C,D)} and a system can
easily be copied by calling \code{lss(sys)} where \emph{sys} is a
lss object itself.
\begin{quote}\begin{description}
\item[{Parameters}] \leavevmode\begin{itemize}
\item {} 
\textbf{A} (\emph{array\_like}) -- 

\item {} 
\textbf{B} (\emph{array\_like}) -- 

\item {} 
\textbf{C} (\emph{array\_like}) -- 

\item {} 
\textbf{D} (\emph{array\_like}) -- State-Space matrices. If one of the matrices is None, it is
replaced by a zero matrix with appropriate dimensions.

\item {} 
\textbf{reduction} (\emph{\{`truncation\_square\_root'\}, optional}) -- Choose method of reduction. If it isn't provided, matrices are
used without reduction.

\item {} 
\textbf{**reduction\_options} (\emph{dict, optional}) -- The arguments with which the reduction method is called.

\end{itemize}

\end{description}\end{quote}
\index{x0 (pod.lss attribute)}

\begin{fulllineitems}
\phantomsection\label{pod:pod.lss.x0}\pysigline{\bfcode{x0}\strong{ array\_like, optional}}
Initial state. Defaults to the zero state.

\end{fulllineitems}

\index{t0 (pod.lss attribute)}

\begin{fulllineitems}
\phantomsection\label{pod:pod.lss.t0}\pysigline{\bfcode{t0}\strong{ float}}
Initial value for \emph{t}

\end{fulllineitems}

\index{integrator (pod.lss attribute)}

\begin{fulllineitems}
\phantomsection\label{pod:pod.lss.integrator}\pysigline{\bfcode{integrator}\strong{ str}}
Name of the integrator used by \code{scipy.integrate.ode}

\end{fulllineitems}

\index{integrator\_options (pod.lss attribute)}

\begin{fulllineitems}
\phantomsection\label{pod:pod.lss.integrator_options}\pysigline{\bfcode{integrator\_options}\strong{ dict}}
Options for the specified integrator that can be set.

\end{fulllineitems}

\index{reduction\_functions (pod.lss attribute)}

\begin{fulllineitems}
\phantomsection\label{pod:pod.lss.reduction_functions}\pysigline{\bfcode{reduction\_functions}\strong{ dict}}
The functions that can be choosen as an input parameter with the
\emph{reduction} keyword.

\end{fulllineitems}

\index{\_\_call\_\_() (pod.lss method)}

\begin{fulllineitems}
\phantomsection\label{pod:pod.lss.__call__}\pysiglinewithargsret{\bfcode{\_\_call\_\_}}{\emph{times}, \emph{control=None}, \emph{force\_ode\_reset=False}}{}
Get the output at specified times with a provided control

It is possible to only request the output at one particular time
or provide a list of times. If \emph{times} is a sequence, the output
will be a list of \code{nparrays} at these times, otherwise it's just
a single \code{nparray}. However the control can either be specified as a
function or is a constant array over all times.
\begin{quote}\begin{description}
\item[{Parameters}] \leavevmode\begin{itemize}
\item {} 
\textbf{times} (\emph{list or scalar}) -- The output for these timese will be calculated

\item {} 
\textbf{control} (callable \code{control(t, y)} or array\_like, optional) -- If it is specified, it will be overwritten in the attributes.

\item {} 
\textbf{force\_ode\_reset} (\emph{Boolean, optional}) -- If it's called, the ode solver is reset and the current attributes
are used.

\end{itemize}

\end{description}\end{quote}

\end{fulllineitems}

\index{f() (pod.lss method)}

\begin{fulllineitems}
\phantomsection\label{pod:pod.lss.f}\pysiglinewithargsret{\bfcode{f}}{\emph{t}, \emph{y}, \emph{u}}{}
Rhs of the differential equation

\end{fulllineitems}

\index{integrator (pod.lss attribute)}

\begin{fulllineitems}
\pysigline{\bfcode{integrator}\strong{ = `dopri5'}}
\end{fulllineitems}

\index{integrator\_options (pod.lss attribute)}

\begin{fulllineitems}
\pysigline{\bfcode{integrator\_options}\strong{ = \{\}}}
\end{fulllineitems}

\index{reduction\_functions (pod.lss attribute)}

\begin{fulllineitems}
\pysigline{\bfcode{reduction\_functions}\strong{ = \{`truncation\_square\_root': \textless{}function truncation\_square\_root at 0x107713cf8\textgreater{}\}}}
\end{fulllineitems}

\index{setupODE() (pod.lss method)}

\begin{fulllineitems}
\phantomsection\label{pod:pod.lss.setupODE}\pysiglinewithargsret{\bfcode{setupODE}}{}{}
Set the ode solver. All integrator, options and initial value can
be set through class attributes.

\end{fulllineitems}

\index{solve() (pod.lss method)}

\begin{fulllineitems}
\phantomsection\label{pod:pod.lss.solve}\pysiglinewithargsret{\bfcode{solve}}{\emph{t}}{}
\end{fulllineitems}

\index{t (pod.lss attribute)}

\begin{fulllineitems}
\phantomsection\label{pod:pod.lss.t}\pysigline{\bfcode{t}}
Current time of the system

\end{fulllineitems}

\index{t0 (pod.lss attribute)}

\begin{fulllineitems}
\pysigline{\bfcode{t0}\strong{ = 0.0}}
\end{fulllineitems}

\index{x (pod.lss attribute)}

\begin{fulllineitems}
\phantomsection\label{pod:pod.lss.x}\pysigline{\bfcode{x}}
State of the system at current time \emph{t}

\end{fulllineitems}

\index{x0 (pod.lss attribute)}

\begin{fulllineitems}
\pysigline{\bfcode{x0}\strong{ = None}}
\end{fulllineitems}

\index{y (pod.lss attribute)}

\begin{fulllineitems}
\phantomsection\label{pod:pod.lss.y}\pysigline{\bfcode{y}}
\end{fulllineitems}


\end{fulllineitems}

\index{truncation\_square\_root() (in module pod)}

\begin{fulllineitems}
\phantomsection\label{pod:pod.truncation_square_root}\pysiglinewithargsret{\code{pod.}\bfcode{truncation\_square\_root}}{\emph{A}, \emph{B}, \emph{C}, \emph{k=0}, \emph{tol=0.0}, \emph{balance=True}, \emph{scale=True}, \emph{check\_stability=True}, \emph{length\_cache\_array=None}}{}
Perform truncation of a system. Scaling and balancing are optional

This allows to reduce a linear state space system by either specifying it
to a certain number of states \emph{k} or by specifying a error tolerance \emph{tol}
relative to the input. In theory the most accurate results are achieved by
using \emph{balance} and \emph{scale} but the size of the error strongly depends on
the particular problem and scaling and balancing may in some cases cost
too much.
\begin{quote}\begin{description}
\item[{Parameters}] \leavevmode\begin{itemize}
\item {} 
\textbf{A} (\emph{array\_like}) -- 

\item {} 
\textbf{B} (\emph{array\_like}) -- 

\item {} 
\textbf{C} (\emph{array\_like}) -- State-Space matrices of the system that should be reduced

\item {} 
\textbf{k} (\emph{int, optional}) -- Order of the output system

\item {} 
\textbf{tol} (\emph{float, optional}) -- Error of the output system, based on the Hankel Singular Values

\item {} 
\textbf{balance} (\emph{Boolean, optional}) -- Balance the system before reducing it to make sure, that the error
is kept small

\item {} 
\textbf{scale} (\emph{Boolean, optional}) -- Scale the system

\item {} 
\textbf{check\_stability} (\emph{Boolean, optional}) -- Checks if all the real parts of the eigenvalues of A are in the left
half of the complex plane.

\end{itemize}

\item[{Returns}] \leavevmode
\begin{itemize}
\item {} 
\textbf{Nr} (\emph{int}) --
Actual size of the system, based on the error: If the Machine error
would have been bigger than the error of the reduced system, it may
happen that \code{Nr \textless{} k} and if the error would be inconsiderably bad,
It might be the case that \code{Nr \textgreater{} k}. In case \emph{k} was never specified,
this is purely based on \emph{tol}

\item {} 
\textbf{Ar, Br, Cr} (\emph{ndarray}) --
Reduced arrays

\item {} 
\textbf{hsv} (\emph{ndarray}) --
Hankel singular values of the original system. The size of the error
may be calculated based on this.

\end{itemize}


\item[{Raises}] \leavevmode\begin{itemize}
\item {} 
\code{ValueError} --
If the system that's provided is not stable (i.e. \emph{A} has eigenvalues
which have non-negative real parts)

\item {} 
\code{ImportError} --
If the slycot subroutine \emph{ab09ad} can't be found. Occurs if the
slycot package is not installed.

\end{itemize}

\end{description}\end{quote}

\end{fulllineitems}



\chapter{tests module}
\label{tests:module-tests}\label{tests:tests-module}\label{tests::doc}\index{tests (module)}
Unit tests and functional tests for the pod.py package
\index{testLss (class in tests)}

\begin{fulllineitems}
\phantomsection\label{tests:tests.testLss}\pysiglinewithargsret{\strong{class }\code{tests.}\bfcode{testLss}}{\emph{methodName='runTest'}}{}
Bases: \code{unittest.case.TestCase}

test lss functionalities
\index{testIdentity() (tests.testLss method)}

\begin{fulllineitems}
\phantomsection\label{tests:tests.testLss.testIdentity}\pysiglinewithargsret{\bfcode{testIdentity}}{}{}
\end{fulllineitems}

\index{test\_abcd\_normalize() (tests.testLss method)}

\begin{fulllineitems}
\phantomsection\label{tests:tests.testLss.test_abcd_normalize}\pysiglinewithargsret{\bfcode{test\_abcd\_normalize}}{}{}
\end{fulllineitems}

\index{test\_f() (tests.testLss method)}

\begin{fulllineitems}
\phantomsection\label{tests:tests.testLss.test_f}\pysiglinewithargsret{\bfcode{test\_f}}{}{}
\end{fulllineitems}

\index{test\_zero\_control() (tests.testLss method)}

\begin{fulllineitems}
\phantomsection\label{tests:tests.testLss.test_zero_control}\pysiglinewithargsret{\bfcode{test\_zero\_control}}{}{}
\end{fulllineitems}


\end{fulllineitems}



\chapter{Indices and tables}
\label{index:indices-and-tables}\begin{itemize}
\item {} 
\emph{genindex}

\item {} 
\emph{modindex}

\item {} 
\emph{search}

\end{itemize}


\renewcommand{\indexname}{Python Module Index}
\begin{theindex}
\def\bigletter#1{{\Large\sffamily#1}\nopagebreak\vspace{1mm}}
\bigletter{a}
\item {\texttt{analysis}}, \pageref{analysis:module-analysis}
\indexspace
\bigletter{d}
\item {\texttt{debug}}, \pageref{debug:module-debug}
\indexspace
\bigletter{e}
\item {\texttt{example2sys}}, \pageref{example2sys:module-example2sys}
\indexspace
\bigletter{f}
\item {\texttt{futurescipy}}, \pageref{futurescipy:module-futurescipy}
\indexspace
\bigletter{p}
\item {\texttt{pod}}, \pageref{pod:module-pod}
\indexspace
\bigletter{t}
\item {\texttt{tests}}, \pageref{tests:module-tests}
\end{theindex}

\renewcommand{\indexname}{Index}
\printindex
\end{document}
